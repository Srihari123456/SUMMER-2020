\renewcommand{\vec}[1]{\mathbf{#1}}
\begin{enumerate}
\item
The figure for a circle obtained in the question looks like Fig. \ref{fig:stepthreetex}, with radius $\vec{r}$, center $\vec{O}$ and equal chords $\vec{AB}$ and $\vec{CD}$ whose point of intersection is $\vec{X}$.  
\begin{figure}[!ht]
\centering
\resizebox{\columnwidth}{!}{\input{./figs/step3.tex}}
\caption{Circle by Latex-Tikz}
\label{fig:stepthreetex}	
\end{figure}

\item 
Two chords are said to be equal if their lengths are the same. The length of a chord is given by 2r$\sin{\frac{\theta}{2}}$, where $\vec{r}$ is the radius and  $\theta$ is the angle subtended by the chord at the center of the circle. Thus in a circle, equal chords subtend equal angles at the center.

\item Let us assume that the two equal chords $\vec{AB}$ and $\vec{CD}$ subtend equal angles of $\theta$ = 90$\degree$ at the center of the circle. $\theta_1$ and $\theta_2$ are the angles between x-axis and points $\vec{B}$ and $\vec{D}$ respectively.

%\begin{comment}
The values used for constructing the circle in both Python and \LaTeX{}-Tikz is in Table \ref{table:table1}:\\
\begin{table}[ht]
    \begin{center}
    	\input{./tables/initialtable.tex}
  \caption{To construct circle O}
   \label{table:table1}
   \end{center}	
\end{table}
%\end{comment}


\item Finding the coordinates of various points of Fig. \ref{fig:stepthreetex}, let 
\quad $ \vec{O}= \begin{pmatrix}0\\0\end{pmatrix}$
%  \quad $\vec{A}=\begin{pmatrix}r\cos{\big(\theta _1 + \frac{\pi}{2}\big)}\\r\sin{\big{\theta _1 + \frac{\pi}{2}\big)}\end{pmatrix}$

  \quad $ \vec{B}= \begin{pmatrix}r\cos{\theta _1}\\r\sin{\theta _1}\end{pmatrix}$
  \quad $\vec{A}=\begin{pmatrix}r\cos{\big(\theta _1 + \theta\big)}\\r\sin{\big(\theta _1 + \theta\big)}\end{pmatrix}$

  \quad $ \vec{D}= \begin{pmatrix}r\cos{\theta _2}\\r\sin{\theta _2}\end{pmatrix}$
  \quad $\vec{C}=\begin{pmatrix}r\cos{\big(\theta _2 + \theta\big)}\\r\sin{\big(\theta _2 + \theta\big)}\end{pmatrix}$

\item As $\vec{OM} and \vec{ON}$ are the perpendiculars from the center of the circle to the chord,which are the midpoints of $\vec{AB}$ and $\vec{CD}$, and are hence represented as:\\
  \quad $ \vec{M}$ = $\frac{\vec{A+B}}{2}$\quad
  \quad $ \vec{N}$ = $\frac{\vec{C+D}}{2}$\\ 



%\begin{comment}
The derived values used is in Table \ref{table:table2}:\\
\begin{table}[ht]
    \begin{center}
    	\input{./tables/derivedtable.tex}
  \caption{To construct circle O}
   \label{table:table2}
   \end{center}	
\end{table}
%\end{comment}

















\end{enumerate}

\subsubsection*{To Show:}    We need to prove that $\angle{OXD}$ = $\angle{AXO}$.\\
