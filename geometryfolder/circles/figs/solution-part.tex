\renewcommand{\vec}[1]{\mathbf{#1}}
 \begin{enumerate}


\item First we need to show that a perpendicular from the center of a circle to a chord bisects the chord.\\

Take a circle C of radius $\vec{r}$=2cm, whose center is $\vec{O}$ such that $ \vec{O}$= $\begin{pmatrix}0\\0\end{pmatrix}$. 
$\vec{AB}$ is a chord such that $\vec{OX} \perp \vec{AB}$. We need to show that $\vec{AX}$ = $\vec{BX}$.
\newline
 
\begin{enumerate}
	\item $\triangle OAX \cong \triangle OBX $by RHS rule as: 
	\begin{enumerate}
	\item $\angle{OXA} = \angle{OXB} $ \quad \{90\degree\}
	\item $\vec{OA} = \vec{OB}$ \quad \{radius of the circle\}
	\item $\vec{OX} = \vec{OX}$ \quad \{Common\}
	\end{enumerate}
\end{enumerate}
	Therefore $\vec{AX} = \vec{BX}$\\
	Hence the perpendicular from the center of a circle to a chord bisects the chord.

  
\begin{figure}[!ht]
\centering
\resizebox{\columnwidth}{!}{\input{./figs/step1.tex}}
\caption{Circle by Latex-Tikz}
\label{fig:stepone}	
\end{figure}
\end{enumerate}




\item Next we need to show that equal chords of a circle are equidistant from the center.\\

Take a circle C of radius $\vec{r}$=2cm, whose center is $\vec{O}$ such that $ \vec{O}$= $\begin{pmatrix}0\\0\end{pmatrix}$. 
$\vec{AB}$ and $\vec{CD}$ are chords such that $\vec{OX} \perp \vec{AB}$ and $\vec{OY} \perp \vec{CD}$. We need to show that $\vec{OX}$ = $\vec{OY}$.
\newline
 

\begin{enumerate}
	
	\item Since $\vec{OX} \perp \vec{AB}$, $\vec{AX} =  \vec{BX} = \frac{\vec{AB}}{2}$ \{perpendicular from the center of a circle to a chord bisects the chord. \}\\
	Similarly $\vec{CY} =  \vec{DY} = \frac{\vec{CY}}{2}$\\
	\item As $\vec{AB} =  \vec{CD}$\\
	$\frac{\vec{AB}}{2} = \frac{\vec{CD}}{2}$\\
	Hence $\vec{AX} =  \vec{CY}$\\
	
	\item In $\triangle AOX$ and $\triangle COY $\\
	
	\begin{enumerate}
	\item $\angle{OXA} = \angle{OYC} $ \quad \{90\degree\}
	\item $\vec{OA} = \vec{OC}$ \quad \{radius of the circle\}
	\item $\vec{AX} = \vec{CY}$ \quad \{Common\}
	\end{enumerate}
	Hence  $\triangle AOX \cong \triangle COY $by RHS rule.\\
	Therefore, $\vec{OX}=\vec{OY}$
\end{enumerate}
	Hence the equal chords of a circle are equidistant from the center.\\

  
\begin{figure}[!ht]
\centering
\resizebox{\columnwidth}{!}{\input{./figs/step2.tex}}
\caption{Circle by Latex-Tikz}
\label{fig:steptwo}	
\end{figure}
\end{enumerate}




