\renewcommand{\vec}[1]{\mathbf{#1}}
 \begin{enumerate}


\item First we need to show that $\vec{M}$ and $\vec{N}$ are the midpoints of $\vec{AB}$ and $\vec{CD}$ respectively.
%Take a circle C of radius $\vec{r}$=2cm, whose center is $\vec{O}$ such that $ \vec{O}$= $\begin{pmatrix}0\\0\end{pmatrix}$. 
In the given circle of center $\vec{O}$, let $\vec{AB}$ be an arbitrary chord such that $\vec{OX} \perp \vec{AB}$. We need to show that $\vec{AX}$ = $\vec{BX}$.
\newline
 
\begin{enumerate}
	\item $\triangle OAX \cong \triangle OBX $by RHS rule as: 
	\begin{enumerate}
	\item $\angle{OXA} = \angle{OXB} $ \quad \{90\degree\}
	\item $\vec{OA} = \vec{OB}$ \quad \{radius of the circle\}
	\item $\vec{OX} = \vec{OX}$ \quad \{Common\}
	\end{enumerate}
\end{enumerate}
\begin{equation}
	Therefore \vec{AX} = \vec{BX} \label{eq:1}
\end{equation}
	Hence the perpendicular from the center of a circle to a chord bisects the chord.
  
\begin{figure}[!ht]
\centering
\resizebox{\columnwidth}{!}{\input{./figs/step1.tex}}
\caption{Perpendicular from the center of a circle to a chord bisects the chord}
\label{fig:steponetex}	
\end{figure}




\item Next we need to show that equal chords of a circle are equidistant from the center.\\

In the circle given in Fig:\ref{fig:steptwotex} whose center is $\vec{O}$, let$\vec{AB}$ and $\vec{CD}$ be two arbitrary chords such that $\vec{OX} \perp \vec{AB}$ and $\vec{OY} \perp \vec{CD}$. We need to show that $\vec{OX}$ = $\vec{OY}$.
%\newline
 

\begin{enumerate}
	
	\item Since $\vec{OX} \perp \vec{AB}$, $\vec{AX} =  \vec{BX} = \frac{\vec{AB}}{2}$ \{From:\eqref{eq:1} \}\\
	Similarly $\vec{CY} =  \vec{DY} = \frac{\vec{CY}}{2}$\\
	\item As $\vec{AB} =  \vec{CD}$\\
	$\frac{\vec{AB}}{2} = \frac{\vec{CD}}{2}$\\
	Hence 
	\begin{equation}
	\vec{AX} =  \vec{CY} \label{eq:2}
	\end{equation}
	
	\item In $\triangle AOX$ and $\triangle COY $ as shown in Fig.\ref{fig:steptwotex}\\
	
	\begin{enumerate}
	\item $\angle{OXA} = \angle{OYC} $ \quad \{90\degree\}
	\item $\vec{OA} = \vec{OC}$ \quad \{radius of the circle\}
	\item $\vec{AX} = \vec{CY}$ \quad \{From:\eqref{eq:2}\}
	\end{enumerate}
	Hence  $\triangle AOX \cong \triangle COY $ by RHS rule.\\
	\begin{equation}
	Therefore, \vec{OX}=\vec{OY} \label{eq:3}
	\end{equation}
\end{enumerate}
	Hence the equal chords of a circle are equidistant from the center.\\

  
\begin{figure}[!ht]
\centering
\resizebox{\columnwidth}{!}{\input{./figs/step2.tex}}
\caption{Equal chords of a circle are equidistant from the center}
\label{fig:steptwotex}	
\end{figure}


The  following Python code generates Fig. \ref{fig:stepthree}\\
    \begin{lstlisting}
        ./codes/step3.py
    \end{lstlisting}

    \begin{figure}[!ht]
	\centering
	\includegraphics[width=\columnwidth]{./figs/step3.pdf}
	\caption{Circle generated using python}
	\label{fig:stepthree}
    \end{figure}
      The equivalent \LaTeX{}- tikz code generating Fig. \ref{fig:stepthreetex} is
    \begin{lstlisting}
        ./figs/stepthree.tex
    \end{lstlisting}
	
	The above \LaTeX{} code can be compiled as a standalone document as
	\begin{lstlisting}
	./figs/step3_standalone.tex
	\end{lstlisting}


\item In the circle given in Fig.\ref{fig:stepthreetex} we need to show that $\angle{OXA}$ = $\angle{OXD}$. We draw $\vec{OM} \perp \vec{AB}$ and $\vec{ON} \perp \vec{CD}$.\\
In $\triangle OMX$ and $\triangle ONX$,
\begin{enumerate}
	\item $\angle{OMX} = \angle{ONX} $ \quad \{90\degree\}
	\item $\vec{OM} = \vec{ON}$ \quad \{From:\eqref{eq:3}\}
	\item $\vec{OX} = \vec{OX}$ \quad \{Common\}
\end{enumerate}
Hence  $\triangle OMX \cong \triangle ONX $ by RHS rule.\\
	Therefore, $\angle{OXM} = \angle{OXN}$\\
$	\implies \angle{OXA} = \angle{OXD}$\\

Hence Proved.





\end{enumerate}




