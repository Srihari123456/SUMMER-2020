\renewcommand{\vec}[1]{\mathbf{#1}}
\renewcommand{\theequation}{\theenumi}
\begin{enumerate}[label=\thesection.\arabic*.,ref=\thesection.\theenumi]
\numberwithin{equation}{enumi}

\item
The figure for a circle obtained in the question looks like Fig. \ref{fig:stepthreetex}, with radius $\vec{r}$, center $\vec{O}$ and equal chords $\vec{AB}$ and $\vec{CD}$ whose point of intersection is $\vec{X}$.  
\begin{figure}[!ht]
\centering
\resizebox{\columnwidth}{!}{\input{./figs/step3.tex}}
\caption{Circle by Latex-Tikz}
\label{fig:stepthreetex}	
\end{figure}

\item 
Two chords $\vec{AB}$ and $\vec{CD}$ are said to be equal if their lengths are the same.Hence we need to show that $\norm{\vec{B}-\vec{A}} = \norm{\vec{D}-\vec{C}}$ to say the two chords are equal.
\item
Let $\theta_1$, $\theta_2$, $\theta_3$ and $\theta_4$ are the angles between x-axis and points $\vec{B}$,$\vec{A}$ ,$\vec{D}$  and $\vec{C}$ respectively.

%\begin{comment}
The values used for constructing the circle in both Python and \LaTeX{}-Tikz is in Table \ref{table:table1}:\\
\begin{table}[ht]
    \begin{center}
    	\input{./tables/initialtable.tex}
  \caption{To construct circle O}
   \label{table:table1}
   \end{center}	
\end{table}
%\end{comment}


\item Finding the coordinates of various points of Fig. \ref{fig:stepthreetex}, let 
\quad $ \vec{O}= \myvec{0\\0}$
%  \quad $\vec{A}=\begin{pmatrix}r\cos{\big(\theta _1 + \frac{\pi}{2}\big)}\\r\sin{\big{\theta _1 + \frac{\pi}{2}\big)}\end{pmatrix}$

  \quad $ \vec{B}= \myvec{r\cos{\theta _1}\\r\sin{\theta _1}}$
  \quad $\vec{A}=\myvec{r\cos{\theta _2}\\r\sin{\theta _2}}$

  \quad $ \vec{D}= \myvec{r\cos{\theta _3}\\r\sin{\theta _3}}$
  \quad $\vec{C}=\myvec{r\cos{\theta _4 }\\r\sin{\theta _4}}$

\item As $\vec{OM} and \vec{ON}$ are the perpendiculars from the center of the circle to the chord,which are the midpoints of $\vec{AB}$ and $\vec{CD}$, and are hence represented as:\\
\begin{align}
 \vec{M} = \frac{\vec{A+B}}{2}\\
  \vec{N} = \frac{\vec{C+D}}{2} 
\end{align}

\item To find the point of intersection $\vec{X}$ of the two chords:\\
Equations of $\vec{AB}$ and $\vec{CD}$ is represented as
$\myvec{0.74&-2.72\\0.74&2.72}\vec{x}$   = $\myvec{3.97\\-3.97}$
On solving we get $\vec{x}$ = $\myvec{0\\-1.45}$\\
Thus the coordinates of $\vec{X}$ = $\myvec{0\\-1.45}$\\
%\begin{comment}
The derived values used is in Table \ref{table:table2}:
\begin{table}[ht]
    \begin{center}
    	\input{./tables/derivedtable.tex}
  \caption{To construct circle O}
   \label{table:table2}
   \end{center}	
\end{table}
%\end{comment}


\end{enumerate}

\subsubsection*{To Show:}    We need to prove that $\angle{OXD}$ = $\angle{AXO}$.\\
