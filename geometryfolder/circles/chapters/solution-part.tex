\renewcommand{\vec}[1]{\mathbf{#1}}
\renewcommand{\theequation}{\theenumi}
\begin{enumerate}[label=\thesection.\arabic*.,ref=\thesection.\theenumi]
\numberwithin{equation}{enumi}


\item First we need to show that $\vec{M}$ and $\vec{N}$ are the midpoints of $\vec{AB}$ and $\vec{CD}$ respectively.
In the given circle of center $\vec{O}$, let $\vec{AB}$ be an arbitrary chord such that $\vec{OX} \perp \vec{AB}$. We need to show that $\vec{AX}$ = $\vec{BX}$.
Since $\vec{OX} \perp \vec{AB}$ :\\
%(O-X).T(A-X)=0
$\myvec{\vec{O-X}}^\vec{T}\myvec{\vec{A-X}}$ = 0\\
\begin{align}
\implies \myvec{\vec{-X[0]} & \vec{-X[1]}}\myvec{\vec{A[0]-X[0]}\\\vec{A[1]-X[1]}} = 0
\end{align}
This condition is satisfied only when $ \vec{X}$ = $\frac{\vec{A+B}}{2}$\quad i.e.  $\vec{X}$ must be the midpoint of $\vec{AB}$.

\begin{align}
	\therefore \vec{AX} = \vec{BX} \label{eq:1}
\end{align}
Hence Perpendicular from the center of a circle to a chord bisects the chord.
  
\begin{figure}[!ht]
\centering
\resizebox{\columnwidth}{!}{\input{./figs/step1.tex}}
\caption{Perpendicular from the center of a circle to a chord bisects the chord}
\label{fig:steponetex}	
\end{figure}




\item Next we need to show that equal chords of a circle are equidistant from the center.\\

In the circle given in Fig:\ref{fig:steptwotex} whose center is $\vec{O}$, let $\vec{AB}$ and $\vec{CD}$ be two arbitrary chords such that $\vec{OX} \perp \vec{AB}$ and $\vec{OY} \perp \vec{CD}$. We need to show that $\vec{OX}$ = $\vec{OY}$.
%\newline
 

\begin{enumerate}
	
	\item Since $\vec{OX} \perp \vec{AB}$, $\vec{AX} =  \vec{BX} = \frac{\vec{AB}}{2}$ \{From:\eqref{eq:1} \}\\
	Similarly $\vec{CY} =  \vec{DY} = \frac{\vec{CY}}{2}$\\
	\item As $\vec{AB} =  \vec{CD}$\\
	$\frac{\vec{AB}}{2} = \frac{\vec{CD}}{2}$\\
	Hence 
	\begin{align}
	\vec{AX} =  \vec{CY} \label{eq:2}
	\end{align}
	
	\item In $\triangle AOX$ and $\triangle COY $ as shown in Fig.\ref{fig:steptwotex} using baudhayana theorem
	\begin{align}
	\norm{\vec{O-Y}}^2 + \norm{\vec{Y-C}}^2 = \norm{\vec{O-X}}^2 + \norm{\vec{X-A}}^2
	\end{align}
	Using \eqref{eq:2}, 
	\begin{align}
	\norm{\vec{O-Y}}^2 &= \norm{\vec{O-X}}^2
	\\
	\implies \norm{\vec{O-Y}} &= \norm{\vec{O-X}}
	\end{align}
	
	\begin{align}
	Therefore, \vec{OX}=\vec{OY} \label{eq:3}
	\end{align}
\end{enumerate}
	Hence the equal chords of a circle are equidistant from the center.\\

  
\begin{figure}[!ht]
\centering
\resizebox{\columnwidth}{!}{\input{./figs/step2.tex}}
\caption{Equal chords of a circle are equidistant from the center}
\label{fig:steptwotex}	
\end{figure}


The  following Python code generates Fig. \ref{fig:stepthree}\\
    \begin{lstlisting}
        ./codes/step3.py
    \end{lstlisting}

    \begin{figure}[!ht]
	\centering
	\includegraphics[width=\columnwidth]{./figs/step3.pdf}
	\caption{Circle generated using python}
	\label{fig:stepthree}
    \end{figure}
      The equivalent \LaTeX{}- tikz code generating Fig. \ref{fig:stepthreetex} is
    \begin{lstlisting}
        ./figs/stepthree.tex
    \end{lstlisting}
	
	The above \LaTeX{} code can be compiled as a standalone document as
	\begin{lstlisting}
	./figs/step3_standalone.tex
	\end{lstlisting}


\item In the circle given in Fig.\ref{fig:stepthreetex} we need to show that $\angle{OXA}$ = $\angle{OXD}$. We draw $\vec{OM} \perp \vec{AB}$ and $\vec{ON} \perp \vec{CD}$.\\
In $\triangle OMX$ and $\triangle ONX$,using baudhayana theorem
	\begin{align}
	\norm{\vec{N-O}}^2 + \norm{\vec{X-N}}^2 = \norm{\vec{M-O}}^2 + \norm{\vec{X-M}}^2
	\end{align}
	Using \eqref{eq:3}, $\norm{\vec{X-N}}^2 = \norm{\vec{X-M}}^2$\\
\begin{align}
	\implies \norm{\vec{NX}} = \norm{\vec{MX}} \label{eq:4}
\end{align}
In $\triangle OMX$ and $\triangle ONX$,using cosine formula\\
\begin{align}
\cos{\angle{OXN}} = \frac{  \norm{\vec{NX}}^2 + \norm{\vec{OX}}^2  - \norm{\vec{ON}}^2     }{2\norm{\vec{NX}}\norm{\vec{OX}}}
\end{align}
\begin{align}
\cos{\angle{OXM}} = \frac{  \norm{\vec{MX}}^2 + \norm{\vec{OX}}^2  - \norm{\vec{OM}}^2     }{2\norm{\vec{MX}}\norm{\vec{OX}}}
\end{align}
Using \eqref{eq:4} and \eqref{eq:3} 
\begin{align}
\cos{\phase{OXN}} = \cos{\angle{OXM}}
\\
	Therefore, \angle{OXM} = \angle{OXN}
\end{align}
	$\implies \angle{OXA} = \angle{OXD}$


Hence Proved.





\end{enumerate}




