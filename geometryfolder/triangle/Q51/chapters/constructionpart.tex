\renewcommand{\vec}[1]{\mathbf{#1}}
\renewcommand{\theequation}{\theenumi}
\begin{enumerate}[label=\thesection.\arabic*.,ref=\thesection.\theenumi]
\numberwithin{equation}{enumi}

\item
The figure for a triangle obtained in the question looks like Fig. \ref{fig:similar}, with sides a,b,c, an arbitrary interior point $\vec{O}$ and a point $\vec{D}$ on line $\vec{AO}$.  
\begin{figure}[!ht]
\centering
\resizebox{\columnwidth}{!}{\input{./figs/constructionpic.tex}}
\caption{Triangle by Latex-Tikz}
\label{fig:similar}	
\end{figure}

The values used for constructing the triangles in both Python and \LaTeX{}-Tikz is in Table \ref{table:table1}:\\
\begin{table}[ht]
    \begin{center}
    	\input{./tables/table1/inp.tex}
  \caption{To construct $\triangle ABC$}
   \label{table:table1}
   \end{center}	
\end{table}

\begin{comment}
\begin{table}[ht]
    \centering
    \begin{tabular}{ |p{1cm}|p{3cm}|  }
    \hline
    
    \multicolumn{2}{|c|}{Initial Input Values.} \\
    \hline
    a & 5\\
    \hline
    b & 6\\
    \hline
    c & 4\\
    \hline
    p & $(a^2 + c^2 - b^2)/(2*a)$\\
    \hline
    q & $(\sqrt(c^2 - p^2)$\\
    \hline
    \end{tabular}
    \caption{To construct $\triangle ABC$}
    \label{table:table1}	
\end{table}
\end{comment}

%\item
%The derived value of $\vec{p}$ and $\vec{q}$ is available in Table \ref{table:table2}

\item
Finding the coordinates of various points of Fig. \ref{fig:similar} :\\
From the information provided in the Table \ref{table:table1}: let\\
\begin{multline}
 \vec{B}= \myvec{0\\0}
\vec{C}=\myvec{a\\0}
\vec{A}=\myvec{p\\q}
\end{multline}
    \\The derived value of $\vec{p}$ and $\vec{q}$ is available in Table \ref{table:table2}.
\item Given a point $\vec{O}$, we need to determine whether it lies inside $\triangle ABC$. Consider 3 vectors $\vec{v_1}$, $\vec{v_2}$ and $\vec{v_3}$ which are orthogonal to vectors $\vec{AB}$,$\vec{BC}$ and $\vec{CA}$ which are ordered counterclock-wise.
%Let $\vec{A}$ = $\begin{pmatrix}x_1\\y_1\end{pmatrix}$ \quad $\vec{B}$ = $\begin{pmatrix}x_2\\y_2\end{pmatrix}$ \quad $\vec{C}$ = $\begin{pmatrix}x_3\\y_3\end{pmatrix}$\quad $\vec{O}$ = $\begin{pmatrix}x\\y\end{pmatrix}$\\
\begin{align}
\vec{AB} = \vec{B} - \vec{A} \\
%= $\begin{pmatrix}x_2-x_1\\y_2-y_1\end{pmatrix}$\\
\vec{BC} = \vec{C} - \vec{B}\\
% = $\begin{pmatrix}x_3-x_2\\y_3-y_2\end{pmatrix}$\\
\vec{CA} = \vec{A} - \vec{C}
% = $\begin{pmatrix}x_1-x_3\\y_1-y_3\end{pmatrix}$\\
\end{align}
As $\vec{v_1}$ is orthogonal to $\vec{AB}$, dot product of $\vec{v_1}$ and $\vec{AB}$ is 0.
This condition is satisfied when \\
\begin{align}
\vec{v_1}= \myvec{\vec{AB[1]}\\\vec{-AB[0]}}\\
\text{Similarly }\vec{v_2}= \myvec{\vec{BC[1]}\\\vec{-BC[0]}}\\
\vec{v_3}= \myvec{\vec{CA[1]}\\\vec{-CA[0]}}
\end{align}
Position vector of $\vec{O}$ w.r.t $\vec{A}$ is 
\begin{align}
\vec{v_{1}^{'}} = \vec{O-A}
\end{align}
Position vector of $\vec{O}$ w.r.t $\vec{B}$ is
\begin{align}
 \vec{v_{2}^{'}} = \vec{O-B}
 \end{align}
Position vector of $\vec{O}$ w.r.t $\vec{C}$ is
\begin{align}
 \vec{v_{3}^{'}} = \vec{O-C}
 \end{align}
Now we compute the dot products: $\vec{O}$ lies inside $\triangle ABC$ only if $dot_1$ ,$dot_2$ and $dot_3$ are all $\geqslant$ 0, where $dot_1 = v_1 \cdot v_{1}^{'} \quad dot_2 = v_2 \cdot v_{2}^{'} \quad dot_3 = v_3 \cdot v_{3}^{'}$.






%The cross product $\vec{AB} \times \vec{AO}$ is defined as a vector $\vec{n}$ that is perpendicular (orthogonal) to both $\vec{AB}$ and $\vec{AO}$, with a direction given by the right-hand rule. 
   %   A point $\vec{O}$ is said to lie inside $\triangle ABC$ if and only if all of the cross products $\vec{AB} \times \vec{AO}$, 
   %   $\vec{BC} \times \vec{BO}$  and $\vec{CA} \times \vec{CO}$ are $\geqslant$ 0.
      

% A point $\vec{O}$ is said to lie inside $\triangle ABC$ if and only if all of the cross products $\vec{AB} \times \vec{AO}$, $\vec{BC} \times \vec{BO}$  and $\vec{CA} \times \vec{CO}$ point in the same direction relative to the plane. That is, either all of them point out of the plane , or all of them point into the plane. The necessary criteria to satisfy this condition is $\vec{AB} \times \vec{AO}$, $\vec{BC} \times \vec{BO}$  and $\vec{CA} \times \vec{CO}$ must be $\geqslant$ 0.
      % point in the same direction relative to the plane. 
      
    %  \begin{equation}
    %  \textbf{ar\big($\triangle ABC$\big) = ar\big($\triangle AOB$\big) + ar\big($\triangle ACO$\big) + ar\big($\triangle OCB$\big)}
    %  \end{equation}
      
  %  Since $\vec{O}$ is an interior point of $\triangle ABC$ ,
\item   Let the arbitrary interior point $\vec{O}$ be represented as $\myvec{2\\1.5}$.    
    $\vec{D}$ is a point on line $\vec{AO}$ such that $\vec{DE}$ $\parallel$ $\vec{OB}$ \quad and \quad $\vec{DF}$  $\parallel$  $\vec{OC}$.\\
\item Determination of points D,E and F:\\
As $\vec{DE}$ $\parallel$ $\vec{OB}$, by basic proportionality theorem the points $\vec{E}$ and $\vec{D}$, divide the lines $\vec{AB}$ and $\vec{AO}$ respectively in the same ratio.\\ 
Hence we choose points $\vec{E}$ and $\vec{D}$ such that
 \begin{align}\frac{AE}{EB} = \frac{AD}{DO}\end{align} 
 \quad Similarly point $\vec{F}$ is chosen such that the points $\vec{F}$ and $\vec{D}$, divide the lines $\vec{AC}$ and $\vec{AO}$ respectively in the same ratio such that 
 \begin{align}\frac{AF}{FC} = \frac{AD}{DO}\end{align}.\\
  
  
  \begin{table}[ht]
    \begin{center}
    	\input{./tables/table2/inp2.tex}
  \caption{To construct $\triangle ABC$}
   \label{table:table2}
   \end{center}	
\end{table}


\item If the point $\vec{D}$ divides the line $\vec{AO}$ in the ratio x:y, the coordinates of $\vec{D}$ is given by section formula as:
\begin{align} \vec{D} = \frac{y\vec{A} + x\vec{O}}{x+y}\end{align}
Similarly the coordinates of points $\vec{E}$ and $\vec{F}$ is given by
\begin{align} \vec{E} = \frac{y\vec{A} + x\vec{B}}{x+y}\end{align}
\begin{align} \vec{F} = \frac{y\vec{A} + x\vec{C}}{x+y}\end{align}

Let us assume the points divide the respective lines in the ratio 1:1. Then the coordinates of points $\vec{D}$, $\vec{E}$ and $\vec{F}$ is \\
 \quad $ \vec{D}= \myvec{1.25\\2.73}$
 %}\begin{pmatrix}1.25\\2.73\end{pmatrix}$
    \quad $\vec{E}=\myvec{0.25\\1.98}$
    \quad $\vec{F}=\myvec{2.75\\1.98}$\\
    
    
\item To check whether $\vec{D}$ lies on line $\vec{AO}$:\\
% substituting the values of the x and y co-ordinate of $\vec{D}$ must satisfy the equation of line $\vec{AO}$. Equation of line joining two points 
Let
\begin{multline}
 \vec{AD} = \vec{D} - \vec{A}\text{	and}
\vec{AO} = \vec{O} - \vec{A}
\end{multline}
$\vec{D}$ lies on $\vec{AO}$ if the below equation is satisfied:
\begin{align}
\frac{\vec{AD[0]}}{\vec{AD[1]}} = \frac{\vec{AO[0]}}{\vec{AO[1]}}
\end{align}

%$\big(x_1,y_1\big)$ and $\big(x_2,y_2\big)$ is given by \begin{align} \frac{x-x_1}{x_2-x_1} = \frac{y-y_1}{y_2-y_1}  \end{align}
%  \input{inp2.tex}
  
  \begin{comment}
    \begin{table}[ht!]
	\centering
	\begin{tabular}{ |p{1cm}|p{2cm}|  }
	\hline
	\multicolumn{2}{|c|}{Derived Values.} \\
	\hline
	$\vec{p}$ & 0.5\\						
	\hline
	$\vec{q}$ & 3.96862697\\
	\hline
    	\end{tabular}
	\caption{To construct $\triangle ABC$}
	\label{table:table2}
    \end{table}
    \end{comment}
    The  following Python code generates Fig. \ref{fig:triabcpy}\\
    \begin{lstlisting}
        ./codes/similartriangle.py
    \end{lstlisting}

    \begin{figure}[!ht]
	\centering
	\includegraphics[width=\columnwidth]{./figs/similartrianglefig.pdf}
	\caption{Triangle generated using python}
	\label{fig:triabcpy}
    \end{figure}
      The equivalent \LaTeX{}- tikz code generating Fig. \ref{fig:similar} is
    \begin{lstlisting}
        ./figs/constructionpic.tex
    \end{lstlisting}
	
	The above \LaTeX{} code can be compiled as a standalone document as
	\begin{lstlisting}
	./figs/constructionpic_standalone.tex
	\end{lstlisting}
\end{enumerate}
\subsubsection*{To Show:}    We need to prove that $\vec{EF}$ $\parallel$   $\vec{BC}$.\\

