\renewcommand{\vec}[1]{\mathbf{#1}}
%\begin{flushleft}
    
 %   The \LaTeX{}- tikz code is
  %  \begin{lstlisting}
%        constrpic.tex
   % \end{lstlisting}

 %   The below \LaTeX{} code can be compiled as standalone document
  %  \begin{lstlisting}
  %      anattempt.tex
   % \end{lstlisting}
%\end{flushleft}
%\begin{columns}
%S\column{0.5\textwidth}
\begin{enumerate}
\item
The figure for a triangle obtained in the question looks like Fig. \ref{fig:similar}, with sides a,b,c, an arbitrary interior point $\vec{O}$ and a point $\vec{D}$ on line $\vec{AO}$.  
\begin{figure}[!ht]
\centering
\resizebox{\columnwidth}{!}{\input{./figs/constructionpic.tex}}
\caption{Triangle by Latex-Tikz}
\label{fig:similar}	
\end{figure}

The values used for constructing the triangles in both Python and \LaTeX{}-Tikz is given in Table \ref{table:table1}:\\
\begin{table}[ht]
    \begin{center}
    	\input{inp.tex}
  \caption{To construct $\triangle ABC$}
   \label{table:table1}
   \end{center}	
\end{table}

\begin{comment}
\begin{table}[ht]
    \centering
    \begin{tabular}{ |p{1cm}|p{3cm}|  }
    \hline
    
    \multicolumn{2}{|c|}{Initial Input Values.} \\
    \hline
    a & 5\\
    \hline
    b & 6\\
    \hline
    c & 4\\
    \hline
    p & $(a^2 + c^2 - b^2)/(2*a)$\\
    \hline
    q & $(\sqrt(c^2 - p^2)$\\
    \hline
    \end{tabular}
    \caption{To construct $\triangle ABC$}
    \label{table:table1}	
\end{table}
\end{comment}

%\item
%The derived value of $\vec{p}$ and $\vec{q}$ is available in Table \ref{table:table2}

\item
Finding the coordinates of various points of Fig. \ref{fig:similar} :\\
From the information provided in the Table \ref{table:table1}: let\\
    \quad $ \vec{B}= \begin{pmatrix}0\\0\end{pmatrix}$
    \quad $\vec{C}=\begin{pmatrix}a\\0\end{pmatrix}$
    \quad $\vec{A}=\begin{pmatrix}p\\q\end{pmatrix}$\\The derived value of $\vec{p}$ and $\vec{q}$ is available in Table \ref{table:table2}.
\item Given a point $\vec{O}$, we need to determine whether it lies inside $\triangle ABC$. 
      A point $\vec{O}$ is said to lie inside $\triangle ABC$ if and only if all of the cross products $\vec{AB} \times \vec{AO}$, 
      $\vec{BC} \times \vec{BO}$  and $\vec{CA} \times \vec{CO}$ are $\geqslant$ 0.
      % point in the same direction relative to the plane. 
      
    %  \begin{equation}
    %  \textbf{ar\big($\triangle ABC$\big) = ar\big($\triangle AOB$\big) + ar\big($\triangle ACO$\big) + ar\big($\triangle OCB$\big)}
    %  \end{equation}
      
  %  Since $\vec{O}$ is an interior point of $\triangle ABC$ ,
\item   Let the arbitrary interior point $\vec{O}$ be represented as $\begin{pmatrix}2\\1.5\end{pmatrix}$.\quad
    
    $\vec{D}$ is a point on line $\vec{AO}$ such that $\vec{DE}$ $\parallel$ $\vec{OB}$ \quad and \quad $\vec{DF}$  $\parallel$  $\vec{OC}$.\\
\item Determination of points D,E and F:\\
As $\vec{DE}$ $\parallel$ $\vec{OB}$, by basic proportionality theorem the points $\vec{E}$ and $\vec{D}$, divide the lines $\vec{AB}$ and $\vec{AO}$ respectively in the same ratio.\\ Hence we choose points $\vec{E}$ and $\vec{D}$ such that \begin{equation}\frac{AE}{EB} = \frac{AD}{DO}\end{equation} \quad Similarly point $\vec{F}$ is chosen such that the points $\vec{F}$ and $\vec{D}$, divide the lines $\vec{AC}$ and $\vec{AO}$ respectively in the same ratio such that \begin{equation}\frac{AF}{FC} = \frac{AD}{DO}\end{equation}.\\
  
  
  \begin{table}[ht]
    \begin{center}
    	\input{inp2.tex}
  \caption{To construct $\triangle ABC$}
   \label{table:table2}
   \end{center}	
\end{table}


\item If the point $\vec{D}$ divides the line $\vec{AO}$ in the ratio x:y, the coordinates of $\vec{D}$ is given by section formula as:
\begin{equation} \vec{D} = \frac{y\vec{A} + x\vec{O}}{x+y}\end{equation}
Similarly the coordinates of points $\vec{E}$ and $\vec{F}$ is given by
\begin{equation} \vec{E} = \frac{y\vec{A} + x\vec{B}}{x+y}\end{equation}
\begin{equation} \vec{F} = \frac{y\vec{A} + x\vec{C}}{x+y}\end{equation}

Let us assume the points divide the respective lines in the ratio 1:1. Then the coordinates of points $\vec{D}$, $\vec{E}$ and $\vec{F}$ is \\
 \quad $ \vec{D}= \begin{pmatrix}1.25\\2.73\end{pmatrix}$
    \quad $\vec{E}=\begin{pmatrix}0.25\\1.98\end{pmatrix}$
    \quad $\vec{F}=\begin{pmatrix}2.75\\1.98\end{pmatrix}$\\
%  \input{inp2.tex}
  
  \begin{comment}
    \begin{table}[ht!]
	\centering
	\begin{tabular}{ |p{1cm}|p{2cm}|  }
	\hline
	\multicolumn{2}{|c|}{Derived Values.} \\
	\hline
	$\vec{p}$ & 0.5\\						
	\hline
	$\vec{q}$ & 3.96862697\\
	\hline
    	\end{tabular}
	\caption{To construct $\triangle ABC$}
	\label{table:table2}
    \end{table}
    \end{comment}
    The  following Python code generates Fig. \ref{fig:triabcpy}\\
    \begin{lstlisting}
        ./codes/similartriangle.py
    \end{lstlisting}

    \begin{figure}[!ht]
	\centering
	\includegraphics[width=\columnwidth]{./figs/similartrianglefig.pdf}
	\caption{Triangle generated using python}
	\label{fig:triabcpy}
    \end{figure}
      The equivalent \LaTeX{}- tikz code generating Fig. \ref{fig:similar} is
    \begin{lstlisting}
        ./figs/constructionpic.tex
    \end{lstlisting}
	
	The above \LaTeX{} code can be compiled as a standalone document as
	\begin{lstlisting}
	./figs/constructionpic_standalone.tex
	\end{lstlisting}
\end{enumerate}
\subsubsection*{To Show:}    We need to prove that $\vec{EF}$ $\parallel$   $\vec{BC}$.\\

